\chapter{Analýza existujících nástrojů pro tvorbu bannerů}
\label{chap:analysis}

\section{Adobe Photoshop}
Nejznámější nástroj pro úpravy grafiky od firmy Adobe nabízí širokou škálu možností pro vytváření reklamních bannerů.
Poradí si se statickou grafikou i animovanou. Jeho profesionalita však může být pro některé uživatele náročná a nepřehledná.
Možnost převedení do HTML5 v samotném Photoshopu chybí a je proto nutné využít jiný program.

Co se týče automatizovaného generování bannerů, je pro Photoshop dostupné rozšíření Banner Ad Reflow.
To umožní uživateli definovat šablonu rozmístění objektů a pomocí umělé inteligence se pokusí optimálně obsah banneru rozmístit.
Pokud bych bylo potřeba vygenerovat bannery se stejným rozložením pro více produktů, lze tohoto docílit pomocí skriptování.
To ovšem není pro všechny uživatelé vhodný přístup a velice zvyšuje složitost. Photoshop jako takový nenabízí možnost napojení na poskytovatelé reklamních sítí.

\section{Google Web Designer}
Propracovaný desktopový designer pro tvorbu HTML5 reklamních bannerů jak statických, tak animovaných.
S řadou předpřipravených šablon na míru již daných formátů reklamy poskytovaných Googlem je tento návrhář jedním z nejsilnějších
volně dostupných produktů pro tvorbu propagačních materiálů.

Umožňuje navázat na různé prvky banneru události a jednoduše tím udělat banner více interaktivní.
Dále zahrnuje možnost práci v 3D prostoru, nebo přímo textový HTML editor.
Animace lze vytvořit pouhým přetažením objektů po plátně. Upravovat jde délka animace, \enquote{cesta pohybu}, časová funkce apod. 

Navíc program sám kontroluje řadu vlastností, které platforma Google Ads vyžaduje
(velikost souboru, validní HTML, správný URL odkazy atd.). Výslednou tvorbu je schopen přímo importovat do reklamní kampaně.

\section{Creatopy}
Tento online nástroj už více svých pokročilých služeb nabízí pouze v placených verzích. Mezi jeho schopnosti však spadá vytváření HTML5 bannerů,
animovaných bannerů a generování více bannerů najednou. Poslední zmínění způsob tvorby je však limitován pevně dané položky banneru jako logo, titulek,
tlačítko atd. Dále nabízí tvorbu Stories pro Facebook a Instagram. Na všechny možnosti poskytuje předpřipravené šablony,
které stačí upravit pro individuální potřebu.

Online editor nepoužívá pro práci element canvas, ale každý objekt je samostatný HTML element a aplikováním CSS je dosaženo požadovaného vzhledu.
Takže možnosti úprav jsou dané možnostmi CSS, které v dnešní době zvládnou dostatečnou škálu efektů.
Díky vlastnosti funkci drop-shadow dokáže CSS vytvářet stíny, které přesně kopírují obrázek.
Tím, že editor využívá práci přímo s HTML získává dobrou výkonnost, ale ztrácí, pokud si uživatel chce grafiku uživatel vytvořit sám.
Nástroj spoléhá spíše na využití mnoha předpřipravených tvarů, které se následně dají upravovat. Vytvoření banner umí následně zmenšit nebo
zvětšit na ostatní velikosti bannerů.

\section{Bannerflow}
Další prostředek možnosti tvorby bannerů prostřednictvím Internetu. Své služby nabízí v rámci předplatného.
Po zaplacení se uživatelům otevře jeden z nejpokročilejších nástrojů pro tvorbu online inzerce.
Samozřejmostí pro něj je návrh a zpracovaní grafiky. Bannery dokáže automaticky škálovat a obsah rozmístit v rámci libovolných velikostí.
Navíc oproti výše zmíněným službám umožňuje přímé napojení na hlavní poskytovatele reklamních sítí (nezmiňuje však konkrétně jaké) a
automatické nahrávání bannerů do reklamních kampaní. Dále také shromažďuje analytická data, které zpřístupňuje svým klientům přímo v rámci aplikace.

\section{Srovnání}
Ačkoli autor této práce hledal, nikde neuměl najít nástroj, který by umožnil jednoduše do stejné šablony bannerů nahrát zdroj dat textů a obrázků,
jehož výsledkem by byly bannery se stejným rozložením. Tato funkce může být vhodná např. pro e-shopy,
které nabízí velké množství podobných produktů (pračky, sušičky apod.). Pokud by tato možnost byla dostupná,
mohla by drasticky snížit celkový čas tvorby reklamy. 



\endinput