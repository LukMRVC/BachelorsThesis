\chapter{Úvod}
\label{chap:Introduction}
Internetová reklama se rychle rozšířila a v momentální době se stala jednou z nejpoužívanějších. Každoroční investice do online inzercí řadově v miliardách korun tento fakt jen více podporují.
Nejvíce viditelná grafická reklama se vyskytuje převážně ve formě bannerů a videí. Ty musí být vytvořeny ve více rozměrech, aby byly dobře viditelné na různých zařízeních.
V případě prodeje širokého množství produktů, se tvorba takových bannerů stává časově náročná. Zároveň se zde nabízí možnost automatizace tohoto procesu, díky které by se 
časové měřítko tvorby reklamního materiálu zkrátilo o velkou část.

Tato práce v první částí cílí na objasnění toho, co to vlastně reklama je z hlediska marketingu organizace. V návaznosti na to rozebírá různé formy reklamy, jejich efektivitu
a nosiče. Detailněji jsou zde popsány možnosti nasazení reklamy do Internetového prostředí společně s tím, jak monitorovat, zda si inzerce vede dobře nebo špatně.
Následně se práce zaměřuje na problematiku tvorby reklamních bannerů a nasazení do reklamních sítí jako je například Google nebo Facebook.

Druhá část je zaměřena na analýzu aktuálních nástrojů pro tvorbu grafických promo materiálů. Prozkoumává jejich schopnosti automatizace, úprav, animací, technologické zpracování a možnosti
přímého napojení do reklamních sítí.
Ná základě této analýzy vzniká návrh vlastního řešení. V této části jsou vypsány potřebné editační funkce, které by měla výsledná aplikace poskytovat. Podle daných potřeb
se vybírají technologie umožňující tyto vlastnosti co nejjednodušeji a nejlépe implementovat. Nejdůležitější funkce nástroje jsou rozebrány více do hloubky společně s možnostmi,
jak projekt ukládat.
V neposlední řadě práce popisuje vytvořenou aplikaci a způsob, jakým program funguje.


\endinput