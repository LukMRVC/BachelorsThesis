\section{Internetová reklama}\label{sec:online-ad}
Moderní digitální doba nabízí mnoho způsobů, kterými lze komerční sdělení šířit. Jako jedny z hlavních forem se stává grafická (display) reklama,
reklama ve vyhledávání a emailová reklama. Nejčastějšími poskytovateli se stali sociální sítě a vyhledávače.
Obrovské množství denních uživatelů z nich udělali perfektní platformy pro nabízení inzercí.
Obrázky 2 a 3 zobrazují srovnání návštěvnosti 5 největších poskytovatelů online reklamy v České republice.

Co se týče emailové reklamy, považuje se efektivní formu přímého marketingu, která je navíc zdarma.
Firmy mohou komunikovat se svými zákazníky jakožto stálými odběrateli nebo rozesílat hromadné emaily. Legislativně je však nevyžádaná elektronická pošta ošetřena.
Organizace tedy při posílaní emailů si musí být jisté, že je jejich sdělení oprávněné a nebude označeno jako spam.

Reklama ve vyhledávání se stává již placenou, většinou stylem PPC. Zadavatelé se za příplatek dostanou na první pozice výsledku hledání.
Toto jim zajišťuje vyšší míru prokliků a lépe nalákají potenciální zákazníky na svou webovou stránku. 

Grafická reklama (nejčastěji bannery a videa) je silným nástrojem pro rozšíření působnosti značky. Buduje povědomí o značce a
zároveň může lákat na produkt. Obsahuje reklamní sdělení, logo firmy a výzvu k akci (CTA) -- nejčastěji ve formě tlačítka.
Obr. Ukazuje srovnání investic do grafické a vyhledávané reklamy.

    \subsection{Metriky internetových reklam}\label{ssec:online-ad-metrics}
    Jak tato práce již zmiňovala, výhodou online inzercí je jednoduchý monitoring. Na rozdíl od ostatních médií si firmy mohou být více jisté,
    že jejich reklamu cílené osoby opravdu vnímají. Hlavní metrikou reklamy vždy bývá zvýšení prodeje.
    Nicméně tato metrika se již měří obtížně a je spíše výsledkem z níže uvedených následujících ukazatelů.

        \subsubsection{Míra prokliku (CTR)}
        Takzvaná míra prokliku určuje, jak je velká šance že na reklamu někdo zareaguje kliknutím myši. Určuje se poměrem mezi celkovým počtem zobrazení reklamy a
        kliknutím na ni. Její význam spočívá v tom, že dokáže prozradit úspěšnost kampaně z hlediska upoutání pozornosti.

        \subsubsection{Míra okamžitého opuštění (BCR)}
        Tato metrika zaznamenává procento uživatelů, kteří se vlivem reklamy dostali na cílovou webovou stránku, ale neprovedou žádnou další aktivitu a
        typicky z webu odejdou. Vysoká procento BCR nejčastěji indikuje jednu z následujících věcí:
        \begin{itemize}
            \item Cílová webová stránka byla nízké kvality. Např. nezajímavý vzhled, neresponzivnost, nepřístupnost, dlouhá doba načítání atd.
            \item Produkt nebo služba nabízená v reklamě nesouvisí s tím, co člověk hledal.
            \item Osoba na stránce našla vše potřebné a další informace nepotřebuje.
        \end{itemize}

        \subsubsection{Konverzní poměr (CVR)}
        Konverzní poměr je výsledkem osob, kteří skrze reklamu dostali na webovou stránku, dokončili nějakou akci (a stali se tak zákazníky) s
        celkovým počtem zobrazení stránky. Akci, kterou mají uživatelé splnit si firma určí sama. Pro e-shopy toto může znamenat dokončení objednávky,
        pro blogy třeba přehlášení k newsletteru.

        \subsubsection{Návratnost investic (ROI)}
        Inzerování je spjato s náklady a investicí. Ve své nejjednodušší podstatě je poměr nákladů a výdělku výslednou metrikou.
        Výdělkem nemusí být pouze zisk, ale i jiný cíl, který si firma stanoví (zvýšení popularity, uživatelů atd.).

    \subsection{Zobrazení online reklamy}
    Pro dobré využití reklamy je potřebné jej umístit na dobře viditelná místa webových stránek s vysokou návštěvností.
    Těmi se stali převážně sociální sítě typu Facebook, Youtube, Twitter apod. Jelikož tito giganti vlastní i více dalších podobných stránek,
    vznikly reklamní sítě jako Google Ads. To umožnilo vznik partnerských webů, které zobrazují reklamy právě prostřednictvím těchto sítí.
    Na oplátky dostávají partnerské weby část tržby ze zobrazených reklam.

    Zmiňované reklamní sítě mají za úkol dát správnou reklamu na správný web.
    Korektní stránku lze vybrat na základě jejího celkového obsahu, klíčových slov vyhledávání spotřebitele či dokonce z analýzy způsobu prohlížení stránky.
    Inzerenti si kupují \enquote{online prostor} pro svou reklamu. 

    Druhou možností je \enquote{kupovat} si cílové publikum své reklamy, a to za pomocí technologie \emph{RTB}.
    Na začátku si inzerent při tvorbě reklamní kampaně vybere svou cílovou skupinu. Tu specifikuje pomocí různých demografických kritérií apod.
    Ve zjednodušené podobě je průběh RTB znázorněn na obrázku č. XXX

\endinput