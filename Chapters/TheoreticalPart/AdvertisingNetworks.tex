\section{Poskytovatelé reklamních sítí}
\label{sec:networks}
Jak tato práce již zmiňovala, reklamám se nejlépe daří na webových stránkách s vysokou návštěvností, protože se inzerce zobrazí více návštěvníkům.
Tato část si podrobněji rozebere a porovná podmínky online bannerové reklamy od několika různých poskytovatelů. 

    \subsection{Google}
    Tento web se stal králem Internetových vyhledávačů. S průměrem kolem 3,5 miliónů vyhledávacích dotazů za minutu (celosvětově),
    jen potvrzuje své dominantní postavení. Reklamu poskytuje v rámci své služby Google Ads (dříve Google AdWords),
    která nabízí vyhledávací kampaně, obsahové kampaně a video kampaně. Co se týče bannerové reklamy,
    Google podporuje 3 formáty obrázkových souborů (PNG, JPG, GIF) standardních specifikovaných velikostí, které řadí do 4 kategorií:
    \begin{itemize}
        \item čtvercové a obdélníkové,
        \item svislé,
        \item výsledková tabule,
        \item mobilní.
    \end{itemize}

    Lze nahrát archiv ZIP obrázků, který nesmí mít více než 40 souborů celkem a všechny obrázky musí být do max 150 KB. V případě že se jedná o GIF animaci,
    nesmí přesáhnout 30 vteřin. Zároveň v animacích nesmí být blikající efekty a frekvence snímků nesmí překročit 5 snímků za vteřinu.

    Za zmínku také stojí možnost tzv. Responzivní reklamy. Ta funguje tak, že inzerent si nahraje svá loga, produkty,
    doplní titulek a ostatní informace. Potom Google za použití strojového učení sám dynamicky sestavuje a zobrazuje kompletní bannery.
    Ovšem při tomto použití nemá inzerent plnou kontrolu nad výslednou grafikou.

    Ohledně reklamní politiku Googlu, nedovoluje propagaci padělaného zboží, nebezpečných produktů a služeb včetně tabákových výrobků,
    nevhodného a nemorálního obsahu jako např. šikana, zastrašování, vydírání, hackování, falešné dokumenty a plagiátorství. Dále, pokud osoba,
    která chce své produkty inzerovat nesmí zneužívat reklam k obsahu, který slouží pouze k navedení na koncového zákazníka někam jinam,
    shromažďovat uživatelská data k neznámým nebo nekalým účelům nebo se vydávat za někoho jiného. S určitým omezením a ohledem na zákony či věk svých koncových
    uživatelů dovoluje Google reklamy na erotický obsah, alkohol, hazardní hry, politický obsah, finanční služby, zdravotnické služby a léky.

    \subsection{Facebook}
    Facebook společně s Instagramem podporují velké množství obrázkových formátů. Ale pro nejlepší výsledky doporučují JPG nebo PNG.
    Tento poskytovatel nezobrazuje standardní bannery, ale pouze obrázky, u kterých záleží na poměru stran.
    Velikosti a poměr stran obrázků se odvíjí podle toho, kam chceme reklamu mít umístěnou. Podporované poměry stran jsou tyto:
    \begin{itemize}
        \item 1,91:1,
        \item 16:9,
        \item 1:1,
        \item 4:5,
        \item 2:3,
        \item 9:16.
    \end{itemize}

    Každý z poměrů stran je vhodný pro zobrazení na různých místech těchto sociálních sítí (např. Kanál příspěvků, Stories, atd.).
    Kompletní tabulku je možno najít na tomto odkaze: https://www.facebook.com/business/help/682655495435254?id=271710926837064

    Minimální doporučené rozlišení je 1080x1080px. Oproti Googlu může být soubor velký až 30 MB.
    Reklamy na Facebooku a Instagramu musí být v souladu se zásady komunity, musí odkazovat na plně funkční úvodní stránky,
    nesmí obsahovat nezákonné produkty nebo služby, diskriminační postupy, nebezpečné doplňky stravy, tabák a s ním související produkty,
    obsah pro dospělé, dezinformace atd.

    \subsection{Sklik}
    Česká služba Sklik firmy Seznam.cz si v České republice vybudovala podobné postavení jako Google.
    Dle dat firmy Gemius, je Seznam.cz největším vydavatelem online reklamy v České republice. Pro lepší inzerci poskytuje cílení
    např. podle témat partnerských webů, zájmů uživatelů, pohlaví, klíčových slov atd.

    Povolené soubory pro nahrání bannerové reklamy jsou PNG, JPG a GIF. Velikostně se soubory musí vlézt do max. 150 KB a
    rovněž podporuje všechny standardní velikosti bannerů.

    Bannery musí vyplňovat celou reklamní plochu, musí být jasně čitelné, relevantní, obsahově a kvalitativně srozumitelné.
    Také by měli mít textové sdělení -- krátký popis nebo výzva k akci. Reklama nesmí napodobovat, nesmí porušovat autorská práva,
    je zakázáno používat clickbaitové sdělení typu \enquote{Budete v šoku} apod. Reklama na erotický obsah se zobrazuje pouze od 22:00 do 6:00.  

\endinput