\chapter{Možná vylepšení}
\label{chap:future}
Z podstaty řešeného problému toutu prací lze najít spoustu dalších možných vylepšení. V dnešní době, kdy je drtivá většina webových
aplikací je nabízená modelem \emph{SaaS} by se i řešení této práce dalo nadále upravovat podobným směrem.
V této kapitole autor práce popisuje pár možností, jak by se výsledný software mohl dále zdokonalovat a vyvíjet.

\section{Editor}
Samotný editor již nabízí uživateli řadu potřebných funkcí. Budoucí iterace by měli práci ještě usnadit a zlepšit. Jedna z prvních schopností
editoru, kterými by mohl v budoucnu disponovat je seskupování objektů. Další užitečné funkce jsou:
\begin{itemize}
    \item volná kresba,
    \item vložení více různých tvarů,
    \item změna spojení čar,
    \item matematické operace s tvary.
\end{itemize}
Při práci se složitějšími designy, by se právě tyto funkce uživateli hodily.
Následně už složitějším vylepšením by bylo přidání časové osy a animací. Samotná použivá knihovna pro práci s canvasem, Konva.js, umožňuje základní
prvky animací jako je pohyb, škálování, rotace, změna barev apod. Přidání tohoto by také znamenalo možnost exportu do jiného formátu umožňující animace.
Těmi by mohli být formát GIF nebo export jako HTML5 banner.

\section{Backend}
Momentální aplikace je celá řešená pouze na straně prohlížeče uživatele a neposkytuje žadný backend. Přidáním této části k aplikaci se otevírají
další příležitosti ke zdokonalení. Ukládání uživatelských projektů na server, ukládání šablon, nahrávání fotek a obrázků, případně i vytváření celých reklamních 
kampaní. Součástí toho by se také mohla stát autorizace a autentizace uživatelů. Projekty by mohly být komplexnější s možností práce více uživatelů najednou.

\subsection{Sdílené projekty}
Velice využívanou funkcí moderních nástrojů pro tvorbů grafiky se stala víceuživatelská práce a sdílení projektů v reálném čase. Na jednom projektu
může pracovat více uživatelů najednou a všichni vidí prováděné změny v reálném čase, nezávisle na tom jak daleko se uživatelé fyzicky od sebe 
nacházejí. Toto je umožněno nejčasteji díky použití \emph{WebSocketů}. Ty udržují neústálé obousměrné spojení mezi klientem a serverem. Pokud tedy server
ví, že stejný projekt má otevřeno více uživatelů, umí prováděné změny posílat přes všechna spojení.

\subsection{Napojení na reklamní sítě}
Správa reklamních kampaní může být poněkud složitá, pokud jsou různé kampaně rozložené na více reklamích sítí. Ty však většinou poskytují možnost,
jak kampaně exportovat, importovat nebo dokonce celé řídit přes \emph{API}. Stejně jako Google i Facebook a český Sklik nabízejí možnost přístupu
přes programové rozhraní. I toto by mohl nástroj v budoucnu sjednocovat a z jednoho místa řídit více kampaní napříč inzertními systémy, včetně
automatizovaného nahrávání vlastních reklamních materiálů.

\endinput