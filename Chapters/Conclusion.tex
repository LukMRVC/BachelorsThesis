\chapter{Závěr}
\label{chap:conclusion}
Tato práce popsala problematiku inzerce, tvorbu online reklamy a její nasazení. Postupně čtenáře provádí stručně od základů marketingu a firemní identity k tomu, jak
reklamu správně využít, vytvořit a umístit. Příblížila varianty měření úspěšnosti online reklamy prostřednictvím různých metrik. Popisuje, jak vytvořit úspěšnou bannerovou reklamu
společně s nutnými podmínkami při nasazení do různých Internetových reklamních sítí. 
Analyzovala existující nástroje, jejich funkce editace a možnosti automatizace pro vytváření bannerů.

Na základě této analýzy vznikla aplikace, která dokáže vytvářet a pracovat s grafikou. Uživatel si při práci s aplikací dokáže vytvořit nebo upravit téměř vše potřebné.
Pro všechny texty je umožněn výběr jakéholiv otevřeného fontu přímo od Googlu. Na všech bannerech lze pracovat najednou nebo i jednotlivě. Do projektu je možno vložit
libovolné obrázky a vytvářet libovolné tvary. Pozadí může být buďto celkovou výplní, přechodnou nebo texturou. Po vytvoření úvodní šablony stačí nahrát 
CSV soubor a automaticky budou vytvořeny všechny další bannery, které se opět dají editovat. Po ukončení práce umí aplikace všechny vytvořenné bannery exportovat do formátů PNG nebo JPEG tak,
aby splnily požadovanou velikost 150~KB. Aplikace je dostupná v českém i anglickém jazyce. Projekty si lze uložit a následně znovu importovat do editoru.
Shrnuty byly i možnosti následného zdokonalení, které by tento nástroj opětovně posunuly dopředu. 


\endinput